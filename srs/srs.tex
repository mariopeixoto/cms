%Copyright 2014 Jean-Philippe Eisenbarth
%This program is free software: you can
%redistribute it and/or modify it under the terms of the GNU General Public
%License as published by the Free Software Foundation, either version 3 of the
%License, or (at your option) any later version.
%This program is distributed in the hope that it will be useful,but WITHOUT ANY
%WARRANTY; without even the implied warranty of MERCHANTABILITY or FITNESS FOR A
%PARTICULAR PURPOSE. See the GNU General Public License for more details.
%You should have received a copy of the GNU General Public License along with
%this program.  If not, see <http://www.gnu.org/licenses/>.

%Based on the code of Yiannis Lazarides
%http://tex.stackexchange.com/questions/42602/software-requirements-specification-with-latex
%http://tex.stackexchange.com/users/963/yiannis-lazarides
%Also based on the template of Karl E. Wiegers
%http://www.se.rit.edu/~emad/teaching/slides/srs_template_sep14.pdf
%http://karlwiegers.com
\documentclass{scrreprt}
\usepackage{listings}
\usepackage{underscore}
\usepackage[bookmarks=true]{hyperref}
\usepackage[utf8]{inputenc}
\usepackage[english]{babel}
\usepackage{graphicx}
\hypersetup{
    bookmarks=false,    % show bookmarks bar?
    pdftitle={Software Requirement Specification},    % title
    pdfauthor={Jean-Philippe Eisenbarth},                     % author
    pdfsubject={TeX and LaTeX},                        % subject of the document
    pdfkeywords={TeX, LaTeX, graphics, images}, % list of keywords
    colorlinks=true,       % false: boxed links; true: colored links
    linkcolor=blue,       % color of internal links
    citecolor=black,       % color of links to bibliography
    filecolor=black,        % color of file links
    urlcolor=purple,        % color of external links
    linktoc=page            % only page is linked
}%
\def\myversion{1.0 }
\date{}
%\title
\usepackage{hyperref}
\begin{document}

\begin{flushright}
    \rule{16cm}{5pt}\vskip1cm
    \begin{bfseries}
        \Huge{SOFTWARE REQUIREMENTS\\ SPECIFICATION}\\
        \vspace{1.5cm}
        for\\
        \vspace{1.5cm}
        Condo Management System\\
        \vspace{1.5cm}
        \LARGE{Version \myversion draft}\\
        \vspace{1.5cm}
        Prepared by Mhaisdhune, Trupti Vishnu\\
        Peixoto Costa Neto, Mario\\
        Yohannan, Blessy\\
        Zhang, Xuezhu\\
        \vspace{1.5cm}
        Texas State University\\
        \vspace{1.5cm}
        \today\\
    \end{bfseries}
\end{flushright}

\tableofcontents

\chapter{Introduction}

\section{Purpose}
The purpose of this document is to present a detailed description of the Condo Management System, a system designed to manage clients, condos, tenants, units and debts and provide organized records of each tenant debts and emit notices for those debts. It will explain the functions that the manager will be able to access, the interfaces of the system, what the system will do, the constraints under which it must operate and how it will react to external stimuli. This document is intended for both users and developers of the system and will be submitted to the customer for approval.

\section{Problem statement}
A law office works as a representative of several condominium administration companies. These companies may have several hundreds tenants who rents their apartments and it’s very common for those tenants to skip rental and administration payments of the condo. The administration company has its own information system, but the law office only receives a monthly report with the up-to-date debts for all condos and units.

The job of the law office is to look at the reports and find the tenants who are in debt for more than a certain time and notify them that they should contact the law office to settle their debts or expect to be sued. But having several companies as clients - each of them having several hundreds of tenants - can lead to a tremendous amount of labor, it’s just not doable.

Each client will provide a monthly debt report with condo names, units in debt, tenants in debt, and debts information and the law office will use that report to import the data into the database.

Then, the system shall provide a screen to help the law office to easily find who’s in debt, for how long they are in debt and how much they owe. That same interface shall allow the law office to choose the tenants to be notified. After choosing, the system shall generate all the notices and the envelope labels for mailing as PDFs.

After notifying the tenants, they may contact to negotiate a settlement or to pay the full amount. Then, the system shall provide a feature to mark those debts as paid.

\chapter{Overall Description}

\section{Product Perspective}
The Condo Management System is a new web-based application and does not depend or is a part of any other system. It has one actor: the manager. The manager needs a compatible web browser and internet access to access the system.

\section{Software Interfaces}
PostgreSQL will be used as a Database Management System for the Condo Management System.

Compatible web-browsers: Chrome, Firefox and Safari.

\section{Product Functions}

\subsection{Create Client}
\subsubsection{Description}

The manager creates a client (Condo Administration Company), entering information such as name, contact name, contact phone and address.

\subsubsection{Rationale}

This fuction will make the client available to the system in order to create the condominia managed by this client. That will be necessary to import the debts and keep an organized record.

\subsection{Edit Client}
\subsubsection{Description}

The manager edits a client (Condo Administration Company), entering information such as name, contact name, contact phone and address.

\subsubsection{Rationale}

This fuction will make possible to edit the client information whenever the data's changed or it was entered incorrectly. That's necessary to keep an organized record.

\subsection{Remove Client}
\subsubsection{Description}

The manager removes a client (Condo Administration Company) from the system.

\subsubsection{Rationale}

This fuction will make possible to remove the client from the system if the client decide to cancel the contract with the law firm. That's necessary to keep an organized record.

\subsection{Create Condo}
\subsubsection{Description}

\subsubsection{Rationale}

\subsection{Edit Condo}
\subsubsection{Description}

\subsubsection{Rationale}

\subsection{Remove Condo}
\subsubsection{Description}

\subsubsection{Rationale}

\subsection{Create Unit}
\subsubsection{Description}

\subsubsection{Rationale}

\subsection{Edit Unit}
\subsubsection{Description}

\subsubsection{Rationale}

\subsection{Remove Unit}
\subsubsection{Description}

\subsubsection{Rationale}

\subsection{Create Tenant}
\subsubsection{Description}

\subsubsection{Rationale}

\subsection{Edit Tenant}
\subsubsection{Description}

\subsubsection{Rationale}

\subsection{Remove Tenant}
\subsubsection{Description}

\subsubsection{Rationale}

\subsection{Create Debt}
\subsubsection{Description}

\subsubsection{Rationale}

\subsection{Edit Debt}
\subsubsection{Description}

\subsubsection{Rationale}

\subsection{Remove Debt}
\subsubsection{Description}

\subsubsection{Rationale}

\chapter{Other Nonfunctional Requirements}

\section{Performance Requirements}
$<$If there are performance requirements for the product under various
circumstances, state them here and explain their rationale, to help the
developers understand the intent and make suitable design choices. Specify the
timing relationships for real time systems. Make such requirements as specific
as possible. You may need to state performance requirements for individual
functional requirements or features.$>$

\section{Safety Requirements}
$<$Specify those requirements that are concerned with possible loss, damage, or
harm that could result from the use of the product. Define any safeguards or
actions that must be taken, as well as actions that must be prevented. Refer to
any external policies or regulations that state safety issues that affect the
product’s design or use. Define any safety certifications that must be
satisfied.$>$

\section{Security Requirements}
$<$Specify any requirements regarding security or privacy issues surrounding use
of the product or protection of the data used or created by the product. Define
any user identity authentication requirements. Refer to any external policies or
regulations containing security issues that affect the product. Define any
security or privacy certifications that must be satisfied.$>$

\section{Software Quality Attributes}
$<$Specify any additional quality characteristics for the product that will be
important to either the customers or the developers. Some to consider are:
adaptability, availability, correctness, flexibility, interoperability,
maintainability, portability, reliability, reusability, robustness, testability,
and usability. Write these to be specific, quantitative, and verifiable when
possible. At the least, clarify the relative preferences for various attributes,
such as ease of use over ease of learning.$>$

\section{Business Rules}
$<$List any operating principles about the product, such as which individuals or
roles can perform which functions under specific circumstances. These are not
functional requirements in themselves, but they may imply certain functional
requirements to enforce the rules.$>$


\chapter{Other Requirements}
$<$Define any other requirements not covered elsewhere in the SRS. This might
include database requirements, internationalization requirements, legal
requirements, reuse objectives for the project, and so on. Add any new sections
that are pertinent to the project.$>$

\end{document}
